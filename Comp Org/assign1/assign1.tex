\documentclass[a4paper]{article}
%Gummi|065|=)

\usepackage[english]{babel}
\usepackage[utf8]{inputenc}
\usepackage{amsmath}
\usepackage{graphicx}
%\usepackage[colorinlistoftodos]{todonotes}

\title{CS310 Assignment 1}

\author{Joe Montey and Sterling Kohel}

\date{\today}

\begin{document}
\maketitle

\begin{enumerate}
\item 
\begin{enumerate}
\item $ \frac {Clock Rate * Executetion Time}{Instruction Count} = CPI$
\\usepackage[colorinlistoftodos]{todonotes}
\\
P1 $\frac {2*3}{6} = \textbf{1 CPI}$
\\
\\
P2 $\frac {1.5*9}{1} = \textbf{6.75 CPI}$
\\
\\
P3 $\frac {3*12}{9} = \textbf{4 CPI}$
\\
\item $\frac {x*3}{2} = 6.75$
\\
\\
$x*3 = 13.5$
\\
\\
$x = \frac {13.5}{3}$
\\
\\
$x = \textbf{4.5Ghz}$
\item $\frac {3*9}{x} = 4$
\\
\\
$\frac {27}{x} = 4$
\\
\\
$27 = 4x$
\\
\\
$\frac {27}{4} = x$
\\
\\
\textbf{$\frac {27}{4} * 10^9$ Instructions}
\\
\end{enumerate}
\item
\begin{enumerate}
\item $ \frac {Clock Rate}{CPI} = Intructions Per Second $
\\
\\
\textbf{P1} $\frac {1}{.6} = \textbf{1.66}$
\\
\\
P2 $\frac {2}{1.5} = 1.33$
\\
\\
P3 $\frac {2.5}{2} = 1.25$
\\
\item P1 $.6 * 1.2 = .72 CPI$
\\
\\
$.72CPI * 2.38 = 1.72GHz$
\\
\\
P2 $1.5 * 1.2 = 1.8 CPI$
\\
\\
$1.8CPI * 1.9 = 3.42GHz$
\\
\\
P3 $2.0 * 1.2 = 2.4 CPI$
\\
\\
$2.4CPI * 1.78 = 4.28GHz$
\\
\item 
P1 10 seconds * 1 Ghz = $10*10^9$ cycles
\\
\\
$\frac{10*10^9 cycles}{.6} = 16.66*10^9$ Instructions
\\
\\
P2 10 seconds * 2 Ghz = $20*10^9$ cycles
\\
\\
$\frac{20*10^9 cycles}{1.5} = 13.33*10^9$ Instructions
\\
\\
P3 10 seconds * 2.5 Ghz = $25*10^9$ cycles
\\
\\
$\frac{25*10^9 cycles}{2.0} = 12.5*10^9$ Instructions
\\
\\
\end{enumerate}
\item
\begin{enumerate}
\item $\frac {1}{period}=frequency$
\\
\\
$\frac {1}{2ns}=500MHz$
\\ 
\\
Compiler A = $\frac {1.5 * 10^9}{500MHz * 2 Seconds} = 1.5 IPC$
\\
\\
Compiler B = $\frac {2.6 * 10^9}{500MHz * 1.5 Seconds} = 3.47 IPC$
\item $\frac {.6 * 10^9}{500MHz * .8 CPI} = .72 seconds$
\\
\\
Compiler A = $\frac {2 Seconds}{.72 Seconds} = 2.78$
\\
\\
Compiler B = $\frac {1.5 Seconds}{.72 Seconds} = 2.08$
\\
\end{enumerate}
\item
\begin{enumerate}
\item P1 $\frac {2.3GHz}{1 Second} = 2300 MIPS$
\\
\\
P2 $\frac {3.5GHz}{2 Seconds} = 1750 MIPS$
\item P1 $\frac {2.5}{33}$
\\
\\
\textbf{P2 $\frac {3.5}{30}$}
\end{enumerate}
\item
\begin{enumerate}
\item
\begin{tabular}{|c|c|c|} 
\hline 
Instruction Type&Frequency&CPI\\  
\hline
ALU ops&$33.3\%$&$1$\\ 
\hline
New ALU ops&$11.1\%$&$2$\\  
\hline
Loads&$14.4\%$&$3$\\
\hline
Stores&$13.3\%$&$3$\\
\hline 
Branches&$27.6\%$&$4$\\
\hline 
\end{tabular}
\item Yes
\end{enumerate}
\item
\begin{enumerate}
\item MIPS = $\frac {clockrate}{CPI * 10^6}$
\begin{itemize}
\item IC: Instruction Count cancels because the total volume of work has no bearing on the rate at which it can be completed.
\item Clock Rate: Click rate is in the numerator because we have chosen CPI, if IPC was chosen instead it would be multiplication.
\item CPI: Since CPI represents how many clocks are needed to do one instruction it makes sense to divide clock speed by CPI. This is multiplied by $10^6$ so that we get the correct units, \textit{millions} of instructions per second.
\end{itemize}
\item No, as demonstrated in problems 3, 4 and 5 MIPS can be very biased by the instructions being run since each instruction can have a different CPI.
\item No, since they have different instruction sets the comparison would be meaningless since the same program could be using entirely different instructions with different CPI and a different number of total instructions.
\item Yes, if P1 is clocked at 2GHz and executes the program at a rate of 1 CPI and P2 is clocked at 4GHz and executes the program at a rate of 3 CPI, the MIPS rating tells us that...
\\
\\
$P1 = \frac {2GHz}{1 * 10^6} = 2000MIPS$
\\
\\
$P2 = \frac {4GHz}{3 * 10^6} = 1333MIPS$
\\
\\
Assuming there are $100 * 10^9$ instructions P1 will take 50 seconds and P2 will take 75 seconds. This means that the MIPS estimate was correct.
\end{enumerate}
\item
\begin{enumerate}
\item Speedup = $\frac{T}{T\prime}$
\\
\\
Speedup = $\frac{500 seconds}{237.5 seconds}$
\\
\\
Speedup = 2.1
\\
\item T$\prime$ = 237.5 seconds
\\
\item .4 = $\frac{500}{T\prime}$
\\
\\
$T\prime$ = 500 * .4
\\
\\
$T\prime$ = 200 seconds
\\
\\
200 seconds Total - 150 seconds remainder Program P = 50 seconds
\\
\item .8 = $\frac{500}{T\prime}$
\
\\
$T\prime$ = 500 * .2
\\
\\
$T\prime$ = 100 seconds
\\
\\
100 seconds Total - 150 seconds remainder Program P = -50 seconds
\\
\\
Impossible 
\end{enumerate}
\end{enumerate}
\end{document}
